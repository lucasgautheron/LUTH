% !TEX TS-program = pdflatex
% !TEX encoding = UTF-8 Unicode

% This is a simple template for a LaTeX document using the "article" class.
% See "book", "report", "letter" for other types of document.

\documentclass[11pt]{article} % use larger type; default would be 10pt

\usepackage[utf8]{inputenc} % set input encoding (not needed with XeLaTeX)

%%% Examples of Article customizations
% These packages are optional, depending whether you want the features they provide.
% See the LaTeX Companion or other references for full information.

%%% PAGE DIMENSIONS
\usepackage{geometry} % to change the page dimensions
\geometry{a4paper} % or letterpaper (US) or a5paper or....
% \geometry{margin=2in} % for example, change the margins to 2 inches all round
% \geometry{landscape} % set up the page for landscape
%   read geometry.pdf for detailed page layout information

\usepackage{graphicx} % support the \includegraphics command and options

% \usepackage[parfill]{parskip} % Activate to begin paragraphs with an empty line rather than an indent

%%% PACKAGES
\usepackage{booktabs} % for much better looking tables
\usepackage{array} % for better arrays (eg matrices) in maths
\usepackage{paralist} % very flexible & customisable lists (eg. enumerate/itemize, etc.)
\usepackage{verbatim} % adds environment for commenting out blocks of text & for better verbatim
\usepackage{subfig} % make it possible to include more than one captioned figure/table in a single float
\usepackage{amsmath}
\usepackage{amssymb}
\usepackage{feynmp}
\usepackage{feynmp-auto}
\usepackage{float}
\usepackage{slashed}
% These packages are all incorporated in the memoir class to one degree or another...

%%% HEADERS & FOOTERS
\usepackage{fancyhdr} % This should be set AFTER setting up the page geometry
\pagestyle{fancy} % options: empty , plain , fancy
\renewcommand{\headrulewidth}{0pt} % customise the layout...
\lhead{}\chead{}\rhead{}
\lfoot{}\cfoot{\thepage}\rfoot{}

%%% SECTION TITLE APPEARANCE
\usepackage{sectsty}
\allsectionsfont{\sffamily\mdseries\upshape} % (See the fntguide.pdf for font help)
% (This matches ConTeXt defaults)

%%% ToC (table of contents) APPEARANCE
\usepackage[nottoc,notlof,notlot]{tocbibind} % Put the bibliography in the ToC
\usepackage[titles,subfigure]{tocloft} % Alter the style of the Table of Contents
\renewcommand{\cftsecfont}{\rmfamily\mdseries\upshape}
\renewcommand{\cftsecpagefont}{\rmfamily\mdseries\upshape} % No bold!

%%% END Article customizations

%%% The "real" document content comes below...

\title{Capture électronique}
\author{}
\date{} % Activate to display a given date or no date (if empty),
         % otherwise the current date is printed 

\begin{document}
\maketitle

\section{$p+e^-\to n+\nu_e$}

On étudie le processus : 
\begin{figure}[H]
\centering
\begin{fmffile}{capture}
\begin{fmfgraph*}(160,100)
% Note that the size is given in normal parentheses
% instead of curly brackets.
% Define external vertices from bottom to top
\fmfleft{i1,i2}
\fmfright{o1,o2}
\fmf{fermion,label=$p(p_p\sigma_p)$}{i1,v1}
\fmf{fermion,label=$n(p_n\sigma_n)$}{v1,o1}
\fmf{fermion,label=$e(p_e\sigma_e)$}{i2,v2}
\fmf{fermion,label=$\nu_e(p_\nu \sigma_\nu)$}{v2,o2}
\fmf{photon,label=$W(k)$}{v1,v2}
\end{fmfgraph*}
\end{fmffile}
\caption{Diagramme de feynman pour $p+e^- \to n+\nu_e$}
\end{figure}

Dans l'hypothèse $k \ll m_W$, l'élément de matrice vaut :
\begin{equation}
\mathcal{M} \simeq -V_{ud}\dfrac{g_W^2}{8}   \dfrac{1}{m_W^2} \left [  \bar{n}(p_n) \gamma^\mu (g_V-g_A\gamma^5) p(p_p) \bar{\nu}(p_\nu) \gamma_\mu (1-\gamma^5) e(p_e) \right ] 
\end{equation}

Et : 

\begin{multline}
\langle |\mathcal{M}|^2 \rangle = \dfrac{1}{4}\sum_{pol} |\mathcal{M}|^2 = \dfrac{1}{4} \left ( |V_{ud}|\dfrac{g_W^2}{8m_W^2} \right) ^2 \mathrm{Tr} \left [ (\slashed{p_n}+m_n)\gamma^\mu (g_V+g_A \gamma^5)(\slashed{p_p}+m_p)\gamma^\nu (g_V+g_A \gamma^5)\right] \\ \cdot \mathrm{Tr} \left [ (\slashed{p_e}+m_e)\gamma_{\mu}(1-\gamma_5) \slashed{p}_{\nu}\gamma_\nu (1-\gamma_5) \right ]
\end{multline}

Et donc :
\begin{align}
\langle |\mathcal{M}|^2 \rangle = \dfrac{1}{4} \left ( |V_{ud}|\dfrac{g_W^2}{m_W^2} \right) ^2 
[ & (g_A-g_V)^2(p_p \cdot p_\nu)(p_n \cdot p_e) \\
&+ (g_A+g_V)^2(p_p\cdot p_e)(p_n \cdot p_\nu) \\ 
&- (g_V^2-g_A^2)m_n m_p (p_e\cdot p_\nu)]
\end{align}

Fermi golden rule :
\begin{equation}
r = 2\pi (2\pi)^4 \delta^4 (p_f-p_i) \dfrac{1}{E_p E_e E_n E_\nu} \langle |\mathcal{M}|^2 \rangle
\end{equation}

\begin{equation}
\dfrac{1}{\lambda (E_\nu)} \equiv \int \dfrac{d^3 p_p}{(2\pi)^3} \int \dfrac{d^3 p_e}{(2\pi)^3} \int \dfrac{d^3 p_n}{(2\pi)^3} r \left [ (1-f_p(E_p))(1-f_e(E_e))2f_n(E_n)\right ]
\end{equation}

Hypothèses :
\begin{itemize}
\item $|\vec{p_n}-\vec{p_p}| \ll E_\nu$
\item $E_\nu \ll m_n$
\item $p_n \ll m_n$, $p_p \ll m_mp$.
\end{itemize}

Si $Q \equiv M_n-M_p$ alors :

Dès lors :
\begin{equation}
p_p \cdot p_\nu \simeq E_p E_\nu \simeq M_p E_\nu
\end{equation}

Et
\begin{equation}
p_p \cdot p_e \simeq E_p E_\nu \simeq M_p E_e = M_p E_e
\end{equation}




\end{document}
