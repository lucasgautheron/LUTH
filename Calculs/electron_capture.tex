% !TEX TS-program = pdflatex
% !TEX encoding = UTF-8 Unicode

% This is a simple template for a LaTeX document using the "article" class.
% See "book", "report", "letter" for other types of document.

\documentclass[11pt]{article} % use larger type; default would be 10pt

\usepackage[utf8]{inputenc} % set input encoding (not needed with XeLaTeX)

%%% Examples of Article customizations
% These packages are optional, depending whether you want the features they provide.
% See the LaTeX Companion or other references for full information.

%%% PAGE DIMENSIONS
\usepackage{geometry} % to change the page dimensions
\geometry{a4paper} % or letterpaper (US) or a5paper or....
% \geometry{margin=2in} % for example, change the margins to 2 inches all round
% \geometry{landscape} % set up the page for landscape
%   read geometry.pdf for detailed page layout information

\usepackage{graphicx} % support the \includegraphics command and options

% \usepackage[parfill]{parskip} % Activate to begin paragraphs with an empty line rather than an indent

%%% PACKAGES
\usepackage{booktabs} % for much better looking tables
\usepackage{array} % for better arrays (eg matrices) in maths
\usepackage{paralist} % very flexible & customisable lists (eg. enumerate/itemize, etc.)
\usepackage{verbatim} % adds environment for commenting out blocks of text & for better verbatim
\usepackage{subfig} % make it possible to include more than one captioned figure/table in a single float
\usepackage{amsmath}
\usepackage{amssymb}
\usepackage{feynmp}
\usepackage{feynmp-auto}
\usepackage{float}
\usepackage{slashed}
% These packages are all incorporated in the memoir class to one degree or another...

%%% HEADERS & FOOTERS
\usepackage{fancyhdr} % This should be set AFTER setting up the page geometry
\pagestyle{fancy} % options: empty , plain , fancy
\renewcommand{\headrulewidth}{0pt} % customise the layout...
\lhead{}\chead{}\rhead{}
\lfoot{}\cfoot{\thepage}\rfoot{}

%%% SECTION TITLE APPEARANCE
\usepackage{sectsty}
\allsectionsfont{\sffamily\mdseries\upshape} % (See the fntguide.pdf for font help)
% (This matches ConTeXt defaults)

%%% ToC (table of contents) APPEARANCE
\usepackage[nottoc,notlof,notlot]{tocbibind} % Put the bibliography in the ToC
\usepackage[titles,subfigure]{tocloft} % Alter the style of the Table of Contents
\renewcommand{\cftsecfont}{\rmfamily\mdseries\upshape}
\renewcommand{\cftsecpagefont}{\rmfamily\mdseries\upshape} % No bold!

%%% END Article customizations

%%% The "real" document content comes below...

\title{Section-efficace de capture électronique}
\author{}
\date{} % Activate to display a given date or no date (if empty),
         % otherwise the current date is printed 

\begin{document}
\maketitle

\section{$u+e^-\to d+\nu_e$}

\begin{figure}[H]
\centering
\begin{fmffile}{capture}
\begin{fmfgraph*}(160,100)
% Note that the size is given in normal parentheses
% instead of curly brackets.
% Define external vertices from bottom to top
\fmfleft{i1,i2}
\fmfright{o1,o2}
\fmf{fermion,label=$u(p_1\sigma_1)$}{i1,v1}
\fmf{fermion,label=$d(q_1\sigma_1)$}{v1,o1}
\fmf{fermion,label=$e(p_2\sigma_2)$}{i2,v2}
\fmf{fermion,label=$\nu_e(q_2\sigma_2)$}{v2,o2}
\fmf{photon,label=$W(k)$}{v1,v2}
\end{fmfgraph*}
\end{fmffile}
\caption{$k = p_1-p_2=q_1-q_2$}
\end{figure}

Élément de matrice :
\begin{equation}
-i\mathcal{M} = \dfrac{-ig_W}{2\sqrt{2}}  V_{ud} \bar{d}(q_1) \gamma^\mu (1-\gamma^5) u(p_1) \times \dfrac{-i(g_{\mu\nu}-k_{\mu}k_{\nu}/m_W^2)}{(k_1+k_2)^2-m_W^2} \times  \dfrac{-ig_W}{2\sqrt{2}} \bar{\nu}(q_2)  \gamma^\mu (1-\gamma^5 e(p_2)
\end{equation}

\begin{multline}
\mathcal{M} = V_{ud}\dfrac{g_W^2}{8}   \dfrac{1}{(p_1+p_2)^2-m_W^2} \left [ \bar{d}(q_1) \gamma^\mu (1-\gamma^5)u(p_1)  \bar{\nu}(q_2) \gamma_\mu (1-\gamma^5) e(p_2) \right. \\  \left. - \bar{d}(q_1)\gamma^\mu (1-\gamma^5) u(p_1) \bar{\nu}(q_2) \gamma_\nu (1-\gamma^5)  e(p_2) p_{\mu}p^{\nu}  /m_W^2 \right ] 
\end{multline}

\subsection{Approximation de Fermi}

Si $p\ll m_W$ :

\begin{equation}
\mathcal{M} \simeq -V_{ud}\dfrac{g_W^2}{8}   \dfrac{1}{m_W^2} \left [  \bar{d}(q_1) \gamma^\mu (1-\gamma^5) u(p_1) \bar{\nu}(q_2) \gamma_\mu (1-\gamma^5) e(p_2) \right ] 
\end{equation}

On note que, puisque $\bar{d} = d^\dagger \gamma^0$
\begin{equation}
(\bar{d}\gamma^\mu(1-\gamma^5)u)^{*} = u^\dagger (1-\gamma^5)^{\dagger}(\gamma^\mu)^{\dagger} (\gamma^0)^\dagger d
\end{equation}

Et de plus $(\gamma^\mu)^\dagger = \gamma^0 \gamma^\mu \gamma^0$ et $(\gamma^0)^2 = 1$ :
\begin{align}
u^\dagger (1-\gamma^5)^{\dagger}(\gamma^\mu)^{\dagger} (\gamma^0)^\dagger d & = u^\dagger (1-\gamma^5)^{\dagger} \gamma^0 \gamma^\mu d\\
& = u^\dagger \gamma^0 \gamma^0 (1-\gamma^5)^{\dagger} \gamma^0 \gamma^\mu d\\
& = \bar{u} \gamma^0 (1-\gamma^5)^{\dagger} \gamma^0 \gamma^\mu d
\end{align}

%Si bien que pour $\Gamma^\mu \equiv \gamma^{\mu} (1-\gamma^5)$
%\begin{equation}
%|\mathcal{M}|^2 = \left ( |V_{ud}|\dfrac{g_W^2}{8m_W^2} \right) ^2 [\bar{d} \Gamma^{\mu} u] \bar{\nu} \bar{\Gamma}_\mu e] [\bar{u} \bar{\Gamma}^{\nu} d]  [\bar{e} \bar{\Gamma}_\nu \nu]
%\end{equation}

On peut tenir compte de la structure interne du proton et du neutron via les facteurs de forme $g_A$ et $g_V$ :

\begin{equation}
\mathcal{M} \simeq -V_{ud}\dfrac{g_W^2}{8}   \dfrac{1}{m_W^2} \left [  \bar{n}(q_1) \gamma^\mu (g_V-g_A\gamma^5) p(p_1) \bar{\nu}(q_2) \gamma_\mu (1-\gamma^5) e(p_2) \right ] 
\end{equation}

Et, en sommant sur les polarisations, et en négligeant $m_\nu$ :
\begin{multline}
\dfrac{1}{4}\sum_{pol} |\mathcal{M}|^2 = \dfrac{1}{4} \left ( |V_{ud}|\dfrac{g_W^2}{8m_W^2} \right) ^2 \mathrm{Tr} \left [ (\slashed{q_1}+m_n)\gamma^\mu (g_V+g_A \gamma^5)(\slashed{p_1}+m_p)\gamma^\nu (g_V+g_A \gamma^5)\right] \\ \cdot \mathrm{Tr} \left [ (\slashed{p_2}+m_e)\gamma_{\mu}(1-\gamma_5) \slashed{q_2}\gamma_\nu (1-\gamma_5) \right ]
\end{multline}

Calcul de la trace (1), sachant que les termes suivants s'annulent :

\begin{itemize}
\item Termes avec un nombre impair de $\gamma^{..}$
\item Termes avec un $\gamma^5$ et un nombre $\leq$ 3 de $\gamma^{..}$.
\end{itemize}


\begin{align}
\mathrm{Tr} \left [ (\slashed{q_1}+m_n)\gamma^\mu (g_V+g_A \gamma^5)(\slashed{p_1}+m_p)\gamma^\nu (g_V+g_A \gamma^5)\right] & =  \\
g_V^2 p_{1\alpha}q_{2\beta}\textrm{Tr} (\gamma^{\alpha}\gamma^\mu \gamma^{\beta} \gamma^{\nu})
& + \\
-4ig_Vg_A\epsilon^{\alpha\mu\beta\nu} q_{1\alpha} p_{1\beta} & + \\
-4ig_Vg_A\epsilon^{\beta\nu\alpha\mu} q_{1\beta} p_{1\alpha} & + \\
g_A^2 p_{1\alpha}q_{2\beta}\textrm{Tr} (\gamma^{\alpha}\gamma^\mu \gamma^{\beta} \gamma^{\nu}) & + \\
4 m_n m_p g_V^2 g^{\mu\nu} & + \\
- 4 m_n m_p g_A^2 g^{\mu\nu}
\end{align}

Donc
\begin{align}
\mathrm{Tr} \left [ (\slashed{q_1}+m_n)\gamma^\mu (g_V+g_A \gamma^5)(\slashed{p_1}+m_p)\gamma^\nu (g_V+g_A \gamma^5)\right] & =  \\
(g_V^2+g_A^2) p_{1\alpha} q_{2\beta} \textrm{Tr} (\gamma^{\alpha}\gamma^\mu \gamma^{\beta} \gamma^{\nu}) & + \\
-8ig_Vg_A \epsilon^{\alpha\mu\beta\nu} q_{1\alpha} p_{1\beta} & + \\
4 m_n m_p (g_V^2-g_A^2) g^{\mu\nu}
\end{align}

Et 
\begin{align}
\mathrm{Tr} \left [ (\slashed{p_2}+m_e)\gamma_{\mu}(1-\gamma_5) \slashed{q_2}\gamma_\nu (1-\gamma_5) \right ] & = \\
2p_{2}^{\alpha} q_{2}^\beta \textrm{Tr}(\gamma_\alpha\gamma_\mu\gamma_\beta\gamma_\nu) & + \\
-4i p_{2}^{\alpha} q_{2}^\beta \epsilon_{\alpha\mu\beta\nu} & + \\
-4i p_{2}^{\beta} q_{2}^\alpha \epsilon_{\alpha\nu\beta\mu}
\end{align}

Soit 
\begin{align}
\mathrm{Tr} \left [ (\slashed{p_2}+m_e)\gamma_{\mu}(1-\gamma_5) \slashed{q_2}\gamma_\nu (1-\gamma_5) \right ] & = \\
2p_{2}^{\alpha} q_{2}^\beta \textrm{Tr}(\gamma_\alpha\gamma_\mu\gamma_\beta\gamma_\nu) & + \\
-8i p_{2}^{\alpha} q_{2}^\beta \epsilon_{\alpha\mu\beta\nu}
\end{align}


Section-efficace différentielle :
\begin{equation}
\dfrac{d\sigma}{d\Omega} = \dfrac{1}{64\pi^2 (E_1+E_2)^2} \dfrac{|\vec{q_1}|}{|\vec{p_1}|}  \bar{|\mathcal{M}|^2}
\end{equation}
\end{document}
